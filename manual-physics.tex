\documentclass[11pt,twoside,a4paper]{article}
%{book}

% This is an automatically generated file.
% Do not edit it.
% Changes to this file are not preserved!

\usepackage{tocloft}
\usepackage{hyperref}
\usepackage{listings}
\lstset{
basicstyle=\small\ttfamily,
columns=flexible,
breaklines=true
}
\setlength{\cftsubsecnumwidth}{3.5em}

\title{Manual for Package:
physics\protect\\Revision 1:8M
}
\author{Karl K\"astner}
%\date{}

\begin{document}

\maketitle

\tableofcontents

% licence
% abstract


\section{@Constant}
\subsection{Constant}
${}$
\begin{lstlisting}
 Constant and physical standard quantities

\end{lstlisting}
\subsection{celsius\_to\_kelvin}
${}$
\begin{lstlisting}
 convert temperature from degree Celsius to Kelvin
 function t_K = celsius_to_kelvin(t_C)

\end{lstlisting}
\subsection{depth\_to\_pressure}
${}$
\begin{lstlisting}
 convert depth to pressure in fresh water at standard temperature
 
    z = (p - p0)/(rho g)
 => p = rho g z + p0

 input :
 p0 : nx1 or scalar, pressure at water surface in BAR
 d  : depth in metre

 output :
 p  : nx1, pressure at measurement depth in BAR


\end{lstlisting}
\subsection{kelvin\_to\_celsius}
${}$
\begin{lstlisting}
 convert temperature degree Kelvin to Celsius

\end{lstlisting}
\subsection{optical\_attenuation}
${}$
\begin{lstlisting}

\end{lstlisting}
\subsection{pressure\_to\_depth}
${}$
\begin{lstlisting}
 convert pressure to depth in fresh water at standard temperature
 
 z = (p - p0)/(rho*g)

 input:
 p  : nx1, pressure at measurement depth in BAR
 p0 : nx1 or scalar, pressure at water surface in BAR

 output:
 d  : depth in metre

\end{lstlisting}
\subsection{saturation\_vapor\_pressure}
${}$
\begin{lstlisting}

\end{lstlisting}
\subsection{sound\_absorption\_air}
${}$
\begin{lstlisting}

\end{lstlisting}
\subsection{sound\_absorption\_water}
${}$
\begin{lstlisting}
 sound absrobption in water
 following Francois and Garrison, 1982
 
 
 function alpha = sound_absorption(f,S,D,T)

 input:
 f : frequency (Hz)
 S : salinity
 D : depth (m)
 T : temperature (degree C)

 output:
 alpha = sound attenuation in dB/m (not dB/km)

 function alpha = sound_absorption(f,S,D,T,model)

\end{lstlisting}
\subsection{sound\_velocity\_water}
${}$
\begin{lstlisting}
 sound velocity in water
 following Lubbers and Graaff (1998)
 this formula does not include depth and salinity effects

\end{lstlisting}
\subsection{viscosity\_dynamic\_water}
${}$
\begin{lstlisting}

\end{lstlisting}
\subsection{viscosity\_kinematic\_water}
${}$
\begin{lstlisting}

\end{lstlisting}
\section{physics}
\subsection{beam\_bending\_deflection}
${}$
\begin{lstlisting}

\end{lstlisting}
\subsection{beam\_bending\_moment}
${}$
\begin{lstlisting}

\end{lstlisting}
\subsection{beam\_bending\_strain}
${}$
\begin{lstlisting}

\end{lstlisting}
\subsection{beam\_bending\_stress}
${}$
\begin{lstlisting}

\end{lstlisting}
\subsection{bolt\_stress}
${}$
\begin{lstlisting}

\end{lstlisting}
\subsection{drag\_force}
${}$
\begin{lstlisting}

\end{lstlisting}
\section{hydrogen-spectrum}
\subsection{hydrogen\_spectrum\_1d}
${}$
\begin{lstlisting}

\end{lstlisting}
\subsection{hydrogen\_spectrum\_2012\_12\_02}
${}$
\begin{lstlisting}

\end{lstlisting}
\subsection{hydrogen\_spectrum\_2d}
${}$
\begin{lstlisting}

\end{lstlisting}
\subsection{hydrogen\_spectrum\_3d}
${}$
\begin{lstlisting}

\end{lstlisting}
\section{physics}
\subsection{minimum\_cable\_diameter}
${}$
\begin{lstlisting}

\end{lstlisting}
\subsection{moment\_of\_inertia\_rectangle}
${}$
\begin{lstlisting}

\end{lstlisting}
\subsection{moment\_of\_inertia\_ring}
${}$
\begin{lstlisting}

\end{lstlisting}
\subsection{parabolic\_reflector\_gain}
${}$
\begin{lstlisting}

\end{lstlisting}
\section{salinity}
\subsection{Salinity}
${}$
\begin{lstlisting}

\end{lstlisting}
\subsection{Salinity78}
${}$
\begin{lstlisting}

\end{lstlisting}
\subsection{canter\_cremer\_number}
${}$
\begin{lstlisting}
 Canter Cremer Number
 ratio of fresh water to sea water that flows into the estuary
 Qf : fresh water discharge
 T  : tidal period
 Pt : tidal prism
 Savenije, Salinity and tides, eq. 1.1, 2.35 and 5.67

\end{lstlisting}
\subsection{density2salinity}
${}$
\begin{lstlisting}

\end{lstlisting}
\subsection{dispersion\_hws\_savenije}
${}$
\begin{lstlisting}
 Dispersion at river mouth during high water slack 

 v0 : tidal velocity scale
 E0 : tidal excursion
 h0 : depth
 a  : convergence length
 Nr : Richargson Number

 Savenije 1993c, Savenije, Salinity and Tides, eg. 5.70

\end{lstlisting}
\subsection{dispersion\_tda\_burgh}
${}$
\begin{lstlisting}

\end{lstlisting}
\subsection{richardson\_number}
${}$
\begin{lstlisting}
 Estuarine Richardson Number
 potential energy due to mixing the entire fresh water with sea water
 ratio of potential energy and buoyancy 
 Savenije, Salinity and Tides, 2.36
 drho : difference of sea water and fresh water density
 rho  : fresh water density
 h    : depth
 v    : tidal velocity scale
 N    : Cramer number

\end{lstlisting}
\subsection{salinity}
${}$
\begin{lstlisting}

\end{lstlisting}
\subsection{salinity\_intrusion\_length}
${}$
\begin{lstlisting}

\end{lstlisting}
\subsection{sea\_water\_density}
${}$
\begin{lstlisting}

\end{lstlisting}
\subsection{tidal\_discharge}
${}$
\begin{lstlisting}
 specific tidal discharge (discharge per unit width)
 

\end{lstlisting}
\subsection{tidal\_excursion}
${}$
\begin{lstlisting}
 Tidal excursion length
 Pt : tidal prism
 h0 : depth
 w0 : width

\end{lstlisting}
\subsection{tidal\_prism\_channel}
${}$
\begin{lstlisting}
 Tidal prism
 Pt = int_lsw^hws Q_t dt ~ A E
 z1 : tidal amplitude
 w0 : width of estuary at mouth
 b  : length of width convergence
 dH_dx = rate of damping of H
 c.f. Savenije 2.34, 2.64

\end{lstlisting}
\subsection{tidal\_prism\_estuary}
${}$
\begin{lstlisting}
 Tidal prism
 Pt = int_lsw^hws Q_t dt ~ A E
 z1 : tidal amplitude
 w0 : width of estuary at mouth
 b  : length of width convergence
 dH_dx = rate of damping of H
 c.f. Savenije 2.34, 2.64

\end{lstlisting}
\subsection{tidal\_velocity}
${}$
\begin{lstlisting}

\end{lstlisting}
\section{physics}
\subsection{test\_sound\_absorption\_air}
${}$
\begin{lstlisting}

\end{lstlisting}
\section{turbulence}
\subsection{keps2nu}
${}$
\begin{lstlisting}

\end{lstlisting}
\section{wind-wave}
\subsection{short\_wave\_length}
${}$
\begin{lstlisting}

\end{lstlisting}
\subsection{short\_wave\_shear\_velocity}
${}$
\begin{lstlisting}

\end{lstlisting}
\subsection{wave\_height\_from\_wind\_speed}
${}$
\begin{lstlisting}

\end{lstlisting}
\end{document}
