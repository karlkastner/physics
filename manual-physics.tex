\documentclass[11pt,twoside,a4paper]{article}
%{book}

% This is an automatically generated file.
% Do not edit it.
% Changes to this file are not preserved!

\usepackage{tocloft}
\usepackage{hyperref}
\usepackage{listings}
\lstset{
basicstyle=\small\ttfamily,
columns=flexible,
breaklines=true
}
\setlength{\cftsubsecnumwidth}{3.5em}

\title{Manual for Package:
physics\protect\\Revision 1:7M
}
\author{Karl K\"astner}
%\date{}

\begin{document}

\maketitle

\tableofcontents

% licence
% abstract


\section{@Constant}
\subsection{Constant}
${}$
\begin{lstlisting}
 Constant and physical standard quantities

\end{lstlisting}
\subsection{celsius\_to\_kelvin}
${}$
\begin{lstlisting}
 convert temperature from degree Celsius to Kelvin
 function t_K = celsius_to_kelvin(t_C)

\end{lstlisting}
\subsection{depth\_to\_pressure}
${}$
\begin{lstlisting}
 convert depth to pressure in fresh water at standard temperature
 
    z = (p - p0)/(rho g)
 => p = rho g z + p0

 input :
 p0 : nx1 or scalar, pressure at water surface in BAR
 d  : depth in metre

 output :
 p  : nx1, pressure at measurement depth in BAR


\end{lstlisting}
\subsection{kelvin\_to\_celsius}
${}$
\begin{lstlisting}
 convert temperature degree Kelvin to Celsius

\end{lstlisting}
\subsection{optical\_attenuation}
${}$
\begin{lstlisting}

\end{lstlisting}
\subsection{pressure\_to\_depth}
${}$
\begin{lstlisting}
 convert pressure to depth in fresh water at standard temperature
 
 z = (p - p0)/(rho*g)

 input:
 p  : nx1, pressure at measurement depth in BAR
 p0 : nx1 or scalar, pressure at water surface in BAR

 output:
 d  : depth in metre

\end{lstlisting}
\subsection{saturation\_vapor\_pressure}
${}$
\begin{lstlisting}

\end{lstlisting}
\subsection{sound\_absorption\_air}
${}$
\begin{lstlisting}

\end{lstlisting}
\subsection{sound\_absorption\_water}
${}$
\begin{lstlisting}
 sound absrobption in water
 following Francois and Garrison, 1982
 
 
 function alpha = sound_absorption(f,S,D,T)

 input:
 f : frequency (Hz)
 S : salinity
 D : depth (m)
 T : temperature (degree C)

 output:
 alpha = sound attenuation in dB/m (not dB/km)

 function alpha = sound_absorption(f,S,D,T,model)

\end{lstlisting}
\subsection{sound\_velocity\_water}
${}$
\begin{lstlisting}
 sound velocity in water
 following Lubbers and Graaff (1998)
 this formula does not include depth and salinity effects

\end{lstlisting}
\subsection{viscosity\_dynamic\_water}
${}$
\begin{lstlisting}

\end{lstlisting}
\subsection{viscosity\_kinematic\_water}
${}$
\begin{lstlisting}

\end{lstlisting}
\section{physics}
\subsection{beam\_bending\_deflection}
${}$
\begin{lstlisting}

\end{lstlisting}
\subsection{beam\_bending\_moment}
${}$
\begin{lstlisting}

\end{lstlisting}
\subsection{beam\_bending\_strain}
${}$
\begin{lstlisting}

\end{lstlisting}
\subsection{beam\_bending\_stress}
${}$
\begin{lstlisting}

\end{lstlisting}
\subsection{bolt\_stress}
${}$
\begin{lstlisting}

\end{lstlisting}
\subsection{drag\_force}
${}$
\begin{lstlisting}

\end{lstlisting}
\section{hydrogen-spectrum}
\subsection{hydrogen\_spectrum\_1d}
${}$
\begin{lstlisting}

\end{lstlisting}
\subsection{hydrogen\_spectrum\_2012\_12\_02}
${}$
\begin{lstlisting}

\end{lstlisting}
\subsection{hydrogen\_spectrum\_2d}
${}$
\begin{lstlisting}

\end{lstlisting}
\subsection{hydrogen\_spectrum\_3d}
${}$
\begin{lstlisting}

\end{lstlisting}
\section{physics}
\subsection{minimum\_cable\_diameter}
${}$
\begin{lstlisting}

\end{lstlisting}
\subsection{moment\_of\_inertia\_rectangle}
${}$
\begin{lstlisting}

\end{lstlisting}
\subsection{moment\_of\_inertia\_ring}
${}$
\begin{lstlisting}

\end{lstlisting}
\subsection{parabolic\_reflector\_gain}
${}$
\begin{lstlisting}

\end{lstlisting}
\subsection{test\_sound\_absorption\_air}
${}$
\begin{lstlisting}

\end{lstlisting}
\end{document}
