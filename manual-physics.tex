\documentclass[11pt,twoside,a4paper]{article}
%{book}

% This is an automatically generated file.
% Do not edit it.
% Changes to this file are not preserved!

\usepackage{tocloft}
\usepackage{hyperref}
\usepackage{listings}
\lstset{
basicstyle=\small\ttfamily,
columns=flexible,
breaklines=true
}
\setlength{\cftsubsecnumwidth}{3.5em}

\title{Manual for Package:
physics\protect\\Revision 11M
}
\author{Karl K\"astner}
%\date{}

\begin{document}

\maketitle

\tableofcontents

% licence
% abstract


\section{@Physics}
\subsection{Physics}
${}$
\begin{lstlisting}
 Physics and physical standard quantities

\end{lstlisting}
\subsection{air\_pressure}
${}$
\begin{lstlisting}

\end{lstlisting}
\subsection{beam\_bending\_deflection}
${}$
\begin{lstlisting}

\end{lstlisting}
\subsection{beam\_bending\_moment}
${}$
\begin{lstlisting}

\end{lstlisting}
\subsection{beam\_bending\_strain}
${}$
\begin{lstlisting}

\end{lstlisting}
\subsection{beam\_bending\_stress}
${}$
\begin{lstlisting}

\end{lstlisting}
\subsection{bolt\_stress}
${}$
\begin{lstlisting}

\end{lstlisting}
\subsection{celsius\_to\_kelvin}
${}$
\begin{lstlisting}
 convert temperature from degree Celsius to Kelvin
 function t_K = celsius_to_kelvin(t_C)

\end{lstlisting}
\subsection{depth\_to\_pressure}
${}$
\begin{lstlisting}
 convert depth to pressure in fresh water at standard temperature
 
    z = (p - p0)/(rho g)
 => p = rho g z + p0

 input :
 p0 : nx1 or scalar, pressure at water surface in BAR
 d  : depth in metre

 output :
 p  : nx1, pressure at measurement depth in BAR


\end{lstlisting}
\subsection{drag\_force}
${}$
\begin{lstlisting}

\end{lstlisting}
\subsection{evapotranspiration\_blaney}
${}$
\begin{lstlisting}

\end{lstlisting}
\subsection{heat\_convection\_through\_orifice}
${}$
\begin{lstlisting}

\end{lstlisting}
\subsection{heat\_transfer}
${}$
\begin{lstlisting}

\end{lstlisting}
\subsection{kelvin\_to\_celsius}
${}$
\begin{lstlisting}
 convert temperature degree Kelvin to Celsius

\end{lstlisting}
\subsection{magnetic\_pull\_force}
${}$
\begin{lstlisting}

\end{lstlisting}
\subsection{minimum\_cable\_diameter}
${}$
\begin{lstlisting}

\end{lstlisting}
\subsection{moment\_of\_area}
${}$
\begin{lstlisting}

\end{lstlisting}
\subsection{moment\_of\_inertia\_rectangle}
${}$
\begin{lstlisting}

\end{lstlisting}
\subsection{moment\_of\_inertia\_ring}
${}$
\begin{lstlisting}

\end{lstlisting}
\subsection{optical\_attenuation}
${}$
\begin{lstlisting}

\end{lstlisting}
\subsection{parabolic\_reflector\_gain}
${}$
\begin{lstlisting}

\end{lstlisting}
\subsection{pressure\_to\_depth}
${}$
\begin{lstlisting}
 convert pressure to depth in fresh water at standard temperature
 
 z = (p - p0)/(rho*g)

 input:
 p  : nx1, pressure at measurement depth in BAR
 p0 : nx1 or scalar, pressure at water surface in BAR

 output:
 d  : depth in metre

\end{lstlisting}
\subsection{saturation\_vapor\_pressure}
${}$
\begin{lstlisting}

\end{lstlisting}
\subsection{sound\_absorption\_air}
${}$
\begin{lstlisting}

\end{lstlisting}
\subsection{sound\_absorption\_water}
${}$
\begin{lstlisting}
 sound absrobption in water
 following Francois and Garrison, 1982
 
 
 function alpha = sound_absorption(f,S,D,T)

 input:
 f : frequency (Hz)
 S : salinity
 D : depth (m)
 T : temperature (degree C)

 output:
 alpha = sound attenuation in dB/m (not dB/km)

 function alpha = sound_absorption(f,S,D,T,model)

\end{lstlisting}
\subsection{sound\_velocity\_water}
${}$
\begin{lstlisting}
 sound velocity in water
 following Lubbers and Graaff (1998)
 this formula does not include depth and salinity effects

\end{lstlisting}
\subsection{thermal\_flux}
${}$
\begin{lstlisting}

\end{lstlisting}
\subsection{viscosity\_dynamic\_water}
${}$
\begin{lstlisting}

\end{lstlisting}
\subsection{viscosity\_kinematic\_water}
${}$
\begin{lstlisting}

\end{lstlisting}
\section{acoustics/@Backscatter}
\subsection{Backscatter}
${}$
\begin{lstlisting}
 acoustic backscatter processing

\end{lstlisting}
\subsection{backscatter2ssc}
${}$
\begin{lstlisting}
 convert backscatter to suspended sediment concentration
 c.f lee hanes / sassi, with linear relation for reference concentration

\end{lstlisting}
\subsection{backscatter2ssc\_implicit}
${}$
\begin{lstlisting}
 convert backscatter to suspended sediment concentration

 this is the methog called "implicit" by hanes, though it is here still
 implemented in an explicit way, as "explicit/imlicit" in hanes only
 mean euler forward or trapezoidal integration

\end{lstlisting}
\subsection{backscatter2ssc\_implicit\_sample}
${}$
\begin{lstlisting}
 convert backscatter to suspended sediment concentration, implicit method

\end{lstlisting}
\subsection{backscatter2ssc\_sample}
${}$
\begin{lstlisting}
 convert backscatter 2 suspended sediment concentration

\end{lstlisting}
\subsection{backscatter2ssc\_sassi}
${}$
\begin{lstlisting}
 convert backscatter to suspended sediment concentration
 c.f. sassi

\end{lstlisting}
\subsection{backscatter2ssc\_sassi\_sample}
${}$
\begin{lstlisting}
 convert backscatter to suspended sediment concentration
 c.f. sassi

\end{lstlisting}
\subsection{fit}
${}$
\begin{lstlisting}
 fit backscatter coefficients

 function [res, leverage, w, obj] = fit(obj,ssc0,R0,R,bs,last,param0)

 ssc0		- ns x 1, reference concentration
 R0            - ns x 1, distance to sample along beam
 bs            - ns x nbin, backscatter profile per sample
 R             - ns x nbin, distance to bin from transducer along beam
 last          - last : index last valid bin
 param0        - initial value for parameters

\end{lstlisting}
\subsection{regmat}
${}$
\begin{lstlisting}
 regression matrix

\end{lstlisting}
\section{acoustics/backscatter}
\subsection{attenuation}
${}$
\begin{lstlisting}

 accoustic attenuation coefficient of suspended particles

 c.f hanes 2012

 input
 [d_mm]   = mm            : (sieve) diameter of particles
 [f_hz]   = Hz = 1/s      : transducer frequency
 [C_kgm3] = kg/m^3 = mg/l : mass concentration of sediment

 output
 [a_s]    = 1/m (neper) : total attenuation per unit distance
  a_snu   : viscuous attenuation
  a_ss    : attenuation due to scattering
 for db : chi_db  = 8.7 chi_neper

 for normalization : chi_s = a_s(C_kgm3=2650)

 function [as,asnu,ass,X,chi] = attenuation_coefficient(d_mm,f,C_kgm3,mode)


\end{lstlisting}
\subsection{backscatter\_coefficient}
${}$
\begin{lstlisting}
 analytic determination of the backscatter coefficient

\end{lstlisting}
\subsection{backscatter\_coefficient\_2}
${}$
\begin{lstlisting}
 analytic basckatter coefficient
 thorne 2002
 thorne 2012

\end{lstlisting}
\subsection{backscatter\_form\_function}
${}$
\begin{lstlisting}
 acoustic backscatter form function

 input
 d_mm : particle diameter
 f_Hz : transducer sound frequency
 
 output
 fbs :

\end{lstlisting}
\subsection{backscatter\_to\_concentration}
${}$
\begin{lstlisting}
 convert acoustic backscatter to suspended sediment mass concentration
 backscatter S has to be corrected for attenuation

\end{lstlisting}
\subsection{backscatter\_to\_concentration2}
${}$
\begin{lstlisting}
 convert acoustic backscatter to sediment concentration

\end{lstlisting}
\subsection{calibrate\_backscatter}
${}$
\begin{lstlisting}

\end{lstlisting}
\subsection{derive\_attenuation\_coefficient}
${}$
\begin{lstlisting}

\end{lstlisting}
\subsection{differential\_cross\_section\_geometric}
${}$
\begin{lstlisting}
 differential cross section
 geometrical backscattering for spherical bodies
 ka >> 1, large particles or high frequencies
 k : wave number
 a : radius of the particle

 sigma

\end{lstlisting}
\subsection{intensity\_ratio\_sphere}
${}$
\begin{lstlisting}

\end{lstlisting}
\subsection{normalized\_particle\_radius}
${}$
\begin{lstlisting}
 normalized particle radius

\end{lstlisting}
\subsection{scattering\_cross\_section}
${}$
\begin{lstlisting}

\end{lstlisting}
\subsection{scattering\_cross\_section3}
${}$
\begin{lstlisting}

\end{lstlisting}
\subsection{scattering\_cross\_section\_general}
${}$
\begin{lstlisting}

 acoustic cross sectin ? of sediment particles
 Medwin, ch. 7.5.3
 Axially Symmetric Spherical Mode Solutions

\end{lstlisting}
\subsection{scatterring\_cross\_section\_merckelbach}
${}$
\begin{lstlisting}

\end{lstlisting}
\subsection{sigma\_rayleigh}
${}$
\begin{lstlisting}
 Rayleigh scattering for a sphere (ka << 1)
 small particles or low frequencies
 Medwin 7.5.2 Rayleigh Scatter From a Sphere (ka << 1)

\end{lstlisting}
\subsection{simulate\_backscatter}
${}$
\begin{lstlisting}
 backscatter as it would be measured,
 when radial spreading, near field distortion and attnuation by water have been corrected for,
 i.e. backscatter as caused and attenuated by sediment
 
 output :
   bs : backscatter
  ibs : integral of backscatter from transducer

 bs(R)  = 1/ks(R)^2*C(R)*exp(-int_0^R 2 as(r) dr)
 ibs(R) = int_0^R bs(r) dr

 values are integrated by the midpoint rule

 input :
 dr_m : steps of radial distance from transducer,
        not necessarily along the vertical
 d_mm : sediment diameter
 C_kg : mass concentration of sediment given at mid-points
 f_Hz : sound frequency of transducer

 dimensions:
  	1 (row)    : along range
 	2 (column) : ensemble / profile (space-time)
      3          : grain size class

\end{lstlisting}
\subsection{ssc2backscatter}
${}$
\begin{lstlisting}
 convert suspended sediment concentration to backscatter,
 not including attenuatio by sediment


 function bs = ssc2backscatter(ssc,d_mm,f,varargin)

 input
 d_mm   : particle radius
 f_Hz   : transducer frequency
 C_kgm3 : mass concentration of sediment [ssc] = g/l = kg/m^3

 output
 bs : backscatter, [bs] = (m/s)^2

\end{lstlisting}
\subsection{viscuous\_attenuation}
${}$
\begin{lstlisting}

\end{lstlisting}
\section{acoustics}
\subsection{backscatter\_form\_function\_theoretic}
${}$
\begin{lstlisting}

\end{lstlisting}
\subsection{coherent\_backscatter\_threshold}
${}$
\begin{lstlisting}

\end{lstlisting}
\subsection{sound\_pressure\_sphere\_collision}
${}$
\begin{lstlisting}

\end{lstlisting}
\subsection{sound\_pressure\_to\_db}
${}$
\begin{lstlisting}

\end{lstlisting}
\subsection{sound\_reflection\_water\_surface}
${}$
\begin{lstlisting}

\end{lstlisting}
\subsection{sound\_transmission\_coefficient}
${}$
\begin{lstlisting}

\end{lstlisting}
\section{physics}
\subsection{distance\_two\_horizon}
${}$
\begin{lstlisting}

\end{lstlisting}
\subsection{electrical\_resistance}
${}$
\begin{lstlisting}

\end{lstlisting}
\section{hydrogen-spectrum}
\subsection{hydrogen\_spectrum\_1d}
${}$
\begin{lstlisting}

\end{lstlisting}
\subsection{hydrogen\_spectrum\_2012\_12\_02}
${}$
\begin{lstlisting}

\end{lstlisting}
\subsection{hydrogen\_spectrum\_2d}
${}$
\begin{lstlisting}

\end{lstlisting}
\subsection{hydrogen\_spectrum\_3d}
${}$
\begin{lstlisting}

\end{lstlisting}
\section{hydrology}
\subsection{Weather}
${}$
\begin{lstlisting}

\end{lstlisting}
\subsection{critical\_pressure\_head}
${}$
\begin{lstlisting}

\end{lstlisting}
\subsection{derive\_equilibrium\_soil\_moisture\_profile\_brooks\_corey}
${}$
\begin{lstlisting}

\end{lstlisting}
\subsection{dielectricity\_to\_soil\_moisture}
${}$
\begin{lstlisting}

\end{lstlisting}
\subsection{equilibrium\_soil\_moisture\_profile\_brooks\_corey}
${}$
\begin{lstlisting}

\end{lstlisting}
\subsection{hydraulic\_conductivity\_from\_water\_content}
${}$
\begin{lstlisting}

\end{lstlisting}
\subsection{hydraulic\_conductivity\_genuchten\_from\_pressure}
${}$
\begin{lstlisting}

\end{lstlisting}
\subsection{ice\_bearing\_thickness}
${}$
\begin{lstlisting}

\end{lstlisting}
\subsection{ice\_growth\_thickness}
${}$
\begin{lstlisting}

\end{lstlisting}
\subsection{infiltration\_phillips}
${}$
\begin{lstlisting}

\end{lstlisting}
\subsection{normalized\_water\_content}
${}$
\begin{lstlisting}

\end{lstlisting}
\subsection{open\_water\_evaporation}
${}$
\begin{lstlisting}

\end{lstlisting}
\subsection{potential\_evapotranspiration\_abtew}
${}$
\begin{lstlisting}

\end{lstlisting}
\subsection{potential\_evapotranspiration\_blaney\_criddle}
${}$
\begin{lstlisting}

\end{lstlisting}
\subsection{potential\_evapotranspiration\_langbein}
${}$
\begin{lstlisting}

\end{lstlisting}
\subsection{potential\_evapotranspiration\_makking}
${}$
\begin{lstlisting}

\end{lstlisting}
\subsection{potential\_evapotranspiration\_turc}
${}$
\begin{lstlisting}

\end{lstlisting}
\subsection{simulate\_weather}
${}$
\begin{lstlisting}

\end{lstlisting}
\subsection{soil\_freezing\_depth}
${}$
\begin{lstlisting}

\end{lstlisting}
\subsection{solar\_radiation}
${}$
\begin{lstlisting}

\end{lstlisting}
\section{mechanics}
\subsection{moment\_of\_inertia\_cylinder}
${}$
\begin{lstlisting}

\end{lstlisting}
\subsection{shoreA\_to\_youngs\_modulus}
${}$
\begin{lstlisting}

\end{lstlisting}
\subsection{shoreD\_to\_youngs\_modulus}
${}$
\begin{lstlisting}

\end{lstlisting}
\subsection{strain}
${}$
\begin{lstlisting}

\end{lstlisting}
\section{salinity}
\subsection{Salinity}
${}$
\begin{lstlisting}

\end{lstlisting}
\subsection{Salinity78}
${}$
\begin{lstlisting}

\end{lstlisting}
\subsection{canter\_cremer\_number}
${}$
\begin{lstlisting}
 Canter Cremer Number
 ratio of fresh water to sea water that flows into the estuary
 Qf : fresh water discharge
 T  : tidal period
 Pt : tidal prism
 Savenije, Salinity and tides, eq. 1.1, 2.35 and 5.67

\end{lstlisting}
\subsection{density2salinity}
${}$
\begin{lstlisting}

\end{lstlisting}
\subsection{dispersion\_hws\_savenije}
${}$
\begin{lstlisting}
 Dispersion at river mouth during high water slack 

 v0 : tidal velocity scale
 E0 : tidal excursion
 h0 : depth
 a  : convergence length
 Nr : Richargson Number

 Savenije 1993c, Savenije, Salinity and Tides, eg. 5.70

\end{lstlisting}
\subsection{dispersion\_tda\_burgh}
${}$
\begin{lstlisting}

\end{lstlisting}
\subsection{estuarine\_richardson\_number}
${}$
\begin{lstlisting}

\end{lstlisting}
\subsection{richardson\_number}
${}$
\begin{lstlisting}
 Estuarine Richardson Number
 potential energy due to mixing the entire fresh water with sea water
 ratio of potential energy and buoyancy 
 Savenije, Salinity and Tides, 2.36
 drho : difference of sea water and fresh water density
 rho  : fresh water density
 h    : depth
 v    : tidal velocity scale
 N    : Cramer number

\end{lstlisting}
\subsection{salinity\_dot}
${}$
\begin{lstlisting}

\end{lstlisting}
\subsection{salinity\_from\_dispersion\_savenije}
${}$
\begin{lstlisting}

\end{lstlisting}
\subsection{salinity\_intrusion\_length}
${}$
\begin{lstlisting}

\end{lstlisting}
\subsection{salinity\_ode}
${}$
\begin{lstlisting}

\end{lstlisting}
\subsection{sea\_water\_density}
${}$
\begin{lstlisting}

\end{lstlisting}
\subsection{tidal\_discharge}
${}$
\begin{lstlisting}
 specific tidal discharge (discharge per unit width)
 

\end{lstlisting}
\subsection{tidal\_excursion}
${}$
\begin{lstlisting}
 Tidal excursion length
 Pt : tidal prism
 h0 : depth
 w0 : width

\end{lstlisting}
\subsection{tidal\_prism\_channel}
${}$
\begin{lstlisting}
 Tidal prism
 Pt = int_lsw^hws Q_t dt ~ A E
 z1 : tidal amplitude
 w0 : width of estuary at mouth
 b  : length of width convergence
 dH_dx = rate of damping of H
 c.f. Savenije 2.34, 2.64

\end{lstlisting}
\subsection{tidal\_prism\_estuary}
${}$
\begin{lstlisting}
 Tidal prism
 Pt = int_lsw^hws Q_t dt ~ A E
 z1 : tidal amplitude
 w0 : width of estuary at mouth
 b  : length of width convergence
 dH_dx = rate of damping of H
 c.f. Savenije 2.34, 2.64

\end{lstlisting}
\subsection{tidal\_velocity}
${}$
\begin{lstlisting}

\end{lstlisting}
\section{physics}
\subsection{test\_sound\_absorption\_air}
${}$
\begin{lstlisting}

\end{lstlisting}
\section{turbulence}
\subsection{keps2nu}
${}$
\begin{lstlisting}

\end{lstlisting}
\section{wind-wave}
\subsection{short\_wave\_length}
${}$
\begin{lstlisting}

\end{lstlisting}
\subsection{short\_wave\_shear\_velocity}
${}$
\begin{lstlisting}

\end{lstlisting}
\subsection{wave\_height\_from\_wind\_speed}
${}$
\begin{lstlisting}

\end{lstlisting}
\end{document}
